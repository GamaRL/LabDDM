%% %% %% %%
%%
%% Introducción de la práctica
%%
%% %% %% %%

\documentclass[../main.tex]{subfiles}
\graphicspath{{\subfix{./img/}}}

\begin{document}
\clearpage
\section{Introducción}
\paragraph{Multiplexor.} Un multiplexor es un circuito combinacional que 
selecciona información binaria de una de muchas líneas de entrada y la envía a 
una sola línea de salida. La selección de una línea de entrada dada se 
controla con un conjunto de líneas de selección. Normalmente, hay $2^n$ líneas 
de entrada y $n$ líneas de selección cuyas combinaciones de bits determinan 
cuál entrada se selecciona.

\paragraph{Decodificador.} Un decodificador es un circuito combinacional que 
convierte información binaria de $n$ líneas de entrada a un máximo de $2n$ 
líneas de salida distintas.  Si la información codificada en n bits tiene 
combinaciones que no se usan, el decodificador podría tener menos de
$2n$ salidas.

\paragraph{Full Adder.} Un sumador completo es un circuito combinacional que 
forma la suma aritmética de tres bits.  Tiene tres entradas y dos salidas. Dos 
de las variables de entrada, denotadas por $x$ y $y$, representan los dos bits 
significativos que se sumarán. La tercera entrada, $z$, representa el acarreo 
previo.

Durante el desarrollo de esta práctica, se hace uso de estos componentes con 
la finalidad de solucionar algunos problemas asociados a la construicción de 
algunos circuitos digitales, tales como un sumador serial, una Unidad 
Aritmético-Lógica y un circuitor Desplazador-Rotador.
\end{document}

