%% %% %% %%
%%
%% Conclusiones de la práctica
%%
%% %% %% %%

\documentclass[../main.tex]{subfiles}
\graphicspath{{\subfix{./img/}}}

\begin{document}
\clearpage
\section{Conclusiones}
\paragraph{Ríos Lira, Gamaliel.}
Con la realización de esta práctica, el objetivo planteado al principio de la 
misma se cumplió satisfactoriamente. Por una parte, se utilizaron componentes 
aritméticos para construir tanto el sumador serial como la ALU; multiplexores 
para la ALU y el circuito Desplazador-Rotador; y decodificadores para el 
ejercicio de la misma ALU. La descomposición del sumador en \textit{Full 
Adders} fue de suma imporatancia para constuir varios de los demás 
componentes. Además, con la realización del previo comprendí de mejor manera 
el uso de los multplexores como una herramienta de ``alto nivel'' (ya 
implementada) que nos ayuda a seleccionar una u otra entrada evitando la 
implementación de ``bajo nivel'' del mismo (a través de compuertas lógicas o 
algún método similar).

Una parte que me pareció complicada de esta práctica fue la asignación de 
pines, ya que al ser tantos, cualquier error resultaba en un comportamiento no 
esperado.
%\paragraph{Vélez Grande, Cinthya.}
\end{document}

