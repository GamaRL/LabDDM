%% %% %% %%
%%
%% Conclusiones de la práctica
%%
%% %% %% %%

\documentclass[../main.tex]{subfiles}
\graphicspath{{\subfix{./img/}}}

\begin{document}
\clearpage
\section{Conclusiones}
\paragraph{Ríos Lira, Gamaliel.} Con la realización de la práctica, el 
objetivo de la misma se cumplió de forma satisfactoria. Por un lado, se tuvo 
un acercamiento a las ventajas que trae consigo la utilización de un lenguaje 
de descripción de \textit{hardware} ---tal como \textit{VHDL}--- al comparar 
la velocidad con la que se implementan circuitos con respecto al modo gráfico.  
Y por otro lado, se tuvo una introducción a la forma en la que funciona, 
describiendo sus partes más esenciales: Bibliotecas, Entidades y 
Arquitecturas. De igual forma, todo esto se complementó con la implementación 
en código de algunos ejemplos. La diferencia entre el modelo teórico al que se 
llega a través de álgebra booleana o \textit{Mapas de Kanaugh} es mínimo ya 
que el código que se utiliza es muy similar a la notación algábraica.

\paragraph{Vélez Grande, Cinthya.}
Mediante el desarrollo de la práctica se realizó la implementación de algunos 
circuitos a través del lenguaje VHDL; a partir de ello fue posible comprender 
de qué partes se conforma un código desarrollado en este lenguaje. Aprendimos 
la manera en la que se realiza tanto la declaración de las entidades a 
emplear, como la asignación de las funciones correspondientes a las formas POS 
y SOP, dentro del bloque de arquitectura, de las funciones lógicas de los 
problemas a tratar.
Fue posible visualizar la forma en la que se vincula el desarrollo teórico del 
diseño de los circuitos con su representación en el lenguaje VHDL; obteniendo 
resultados equivalentes tanto en las implementaciones gráficas realizadas de 
manera previa en prácticas anteriores, como en su representación con el 
lenguaje utilizado.
\end{document}

