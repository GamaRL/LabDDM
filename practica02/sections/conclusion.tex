%% %% %% %%
%%
%% Conclusiones de la práctica
%%
%% %% %% %%

\documentclass[../main.tex]{subfiles}
\graphicspath{{\subfix{./img/}}}

\begin{document}
\clearpage
\section{Conclusiones}
\paragraph{Ríos Lira, Gamaliel.} Con la realización de la práctica, se cumplió 
el objtivo planteado al inicio de la misma. Conocimos nuevas funcionalidades 
que nos proporciona el entorno de \textit{Quartus II}, como la creación de 
sub-módulos a través de símbolos, el concepto de vectores y el uso de 
etiquetas para evitar conexiones complejas. Los resultados obtenidos a través 
de la protaforma se comprobaron con éxito a través del análisis previo 
anterior.  Así mismo, ví la importancia de utilizar vectores para facilitar 
las simulaciones y el manejo de las variables. Finalmente, me pareció adecuado 
todo lo que aprendimos para implementar la Parte C de la práctica.

\paragraph{Vélez Grande, Cinthya.} A partir del desarrollo de la práctica fue 
posible implementar diversas funciones booleanas en el entorno de desarrollo 
Quartus II, especificando la interconexión de las variables de entrada con las 
compuertas lógicas que la conforman y las salidas de los sistemas.  Uno de los 
aprendizajes obtenidos fue la forma de abstraer las funciones en forma de 
símbolos que posteriormente se implementaron en otros archivos del tipo BDF 
para obtener en forma de única salida las múltiples salidas generadas; y a 
partir del cronograma generado de la evaluación de las funciones, se comprobó 
que el resultado obtenido era idéntico al obtenido mediante la evaluación 
tabular de las funciones.
\end{document}

