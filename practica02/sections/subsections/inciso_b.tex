%% %% %% %%
%%
%% Parte B de la práctica
%%
%% %% %% %%

\documentclass[../procedimientos.tex]{subfiles}
\graphicspath{{\subfix{../../images/}}}

\begin{document}
\clearpage
\subsection{Parte B}
\subsubsection{Instrucciones}
A partir de las siguientes funciones lógicas, impleméntelas dentro del 
ambiente gráfico de la plataforma \textit{Quatus II} de \textit{Altera}.
\begin{itemize}
  \item $F_1 (xyz) = x (\nt{x} + y\nt{z} + z)$
  \item $F_2 (xyz) = (x + y) (\nt{x} + y) (\nt{y} + \nt{z})$
  \item $F_3 (xyz) = \nt{x}\nt{y} + x\oplus\nt{y}$
  \item $F_4 (xz)  = (x + z)\odot\nt{x}$
\end{itemize}

\subsubsection{Análisis}
De igual forma que en la Parte A, se comenzará haciendo el análsis del 
comportamiento de cada una de las funciones lógicas a través de us tabla de 
verdad. Comenzando con el comportamiento de $f_1(xyz)$.
\begin{table}[H]
  \centering
  \begin{tabular}{ccc|ccc|c}
    \hline
    $x$ & $y$ & $z$ & $\nt{x}$ & $y\nt{z}$ & $\nt{x}+y\nt{z}+z$ & $f_1$\\
    \hline
    0 & 0 & 0 & 1 & 0 & 1 & 0\\
    0 & 0 & 1 & 1 & 0 & 1 & 0\\
    0 & 1 & 0 & 1 & 1 & 1 & 0\\
    0 & 1 & 1 & 1 & 0 & 1 & 0\\
    1 & 0 & 0 & 0 & 0 & 0 & 0\\
    1 & 0 & 1 & 0 & 0 & 1 & 1\\
    1 & 1 & 0 & 0 & 1 & 1 & 1\\
    1 & 1 & 1 & 0 & 0 & 1 & 1\\
    \hline
  \end{tabular}
  \caption{Tabla de verdad de $f_1(xyz)$ (Parte B)}
  \label{tab:b_f1}
\end{table}

Continuando con el comportamiento de $f_2(xyz)$, se tiene que:
\begin{table}[H]
  \centering
  \begin{tabular}{ccc|ccc|c}
    \hline
    $x$ & $y$ & $z$ & $x+y$ & $\nt{x}+y$ & $\nt{y}+\nt{z}$ & $f_2$\\
    \hline
    0 & 0 & 0 & 0 & 1 & 1 & 0\\
    0 & 0 & 1 & 0 & 1 & 1 & 0\\
    0 & 1 & 0 & 1 & 1 & 1 & 1\\
    0 & 1 & 1 & 1 & 1 & 0 & 0\\
    1 & 0 & 0 & 1 & 0 & 1 & 0\\
    1 & 0 & 1 & 1 & 0 & 1 & 0\\
    1 & 1 & 0 & 1 & 1 & 1 & 1\\
    1 & 1 & 1 & 1 & 1 & 0 & 0\\
    \hline
  \end{tabular}
  \caption{Tabla de verdad de $f_2(xyz)$ (Parte B)}
  \label{tab:b_f2}
\end{table}

Se prosigue con el comportamiento de la función lógica $f_3(xyz)$, para esta, 
es primero hacer uso de la equivalencia lógica de la operación XOR, con la 
finalidad de simplificar sus cálculos. Por lo tanto, se obtiene lo siguiente:
\begin{align*}
  f_3(xyz) &= \nt{x}\nt{y} + x\nt{\nt{y}} + \cancel{\nt{x}\nt{y}}\\
  &= \nt{x}\nt{y} + xy
\end{align*}
$$\therefore{f_3(xyz) = \nt{x}\nt{y} + xy}$$

Haciendo uso de la equivalencia anterior, se puede conformar la tabla de 
verdad de la siguiente forma:
\begin{table}[H]
  \centering
  \begin{tabular}{ccc|cc|c}
    \hline
    $x$ & $y$ & $z$ & $\nt{x}\nt{y}$ & $xy$ & $f_3$\\
    \hline
    0 & 0 & 0 & 1 & 0 & 1\\
    0 & 0 & 1 & 1 & 0 & 1\\
    0 & 1 & 0 & 0 & 0 & 0\\
    0 & 1 & 1 & 0 & 0 & 0\\
    1 & 0 & 0 & 0 & 0 & 0\\
    1 & 0 & 1 & 0 & 0 & 0\\
    1 & 1 & 0 & 0 & 1 & 1\\
    1 & 1 & 1 & 0 & 1 & 1\\
    \hline
  \end{tabular}
  \caption{Tabla de verdad de $f_3(xyz)$ (Parte B)}
  \label{tab:b_f3}
\end{table}

Para concluir lo anterior, también se elaborará la tabla de verdad de la 
función $f_4(xy)$, la cual se puede simplificar a través de la equivalencia 
lógica de la compuerta XNOR.
\begin{align*}
  f_4(xz) &= (x + z) \odot \nt{x}\\
  &= \overline{(x+z)}\nt{\nt{x}} + (x+z)x\\
  &= \cancel{(\nt{x}\nt{z})x} + (x+z)\nt{x}\\
  &= \cancel{x\nt{x}} + \nt{x}z\\
  &= \nt{x}z
\end{align*}
$$f_4(xyz) = \nt{x}z$$

Con la equivalencia anterior se puede elaborar la tabla de verdad, tal como se 
muestra a continuación
\begin{table}[H]
  \centering
  \begin{tabular}{ccc|cc|c}
    \hline
    $x$ & $y$ & $\nt{x}z$ & $f_4$\\
    \hline
    0 & 0 & 0 & 0\\
    0 & 1 & 1 & 1\\
    1 & 0 & 0 & 0\\
    1 & 1 & 0 & 0\\
    \hline
  \end{tabular}
  \caption{Tabla de verdad de $f_4(xyz)$ (Parte B)}
  \label{tab:b_f4}
\end{table}

Finalmente, se combinarán los resultados de las Tablas \ref{tab:b_f1}, 
\ref{tab:b_f2}, \ref{tab:b_f3} y \ref{tab:b_f4}, con la finalidad de tener el 
resultado de cada una de ellas en una sola tabla.
\begin{table}[H]
  \centering
  \begin{tabular}{ccc|cccc|c}
    \hline
    $x$ & $y$ & $z$ & $f_1$ & $f_2$ & $f_3$ & $f_4$ & HEX\\
    \hline
    0 & 0 & 0 & 0 & 0 & 1 & 0 & 2\\
    0 & 0 & 1 & 0 & 0 & 1 & 0 & 2\\
    0 & 1 & 0 & 0 & 1 & 0 & 0 & 4\\
    0 & 1 & 1 & 0 & 0 & 0 & 1 & 1\\
    1 & 0 & 0 & 0 & 0 & 0 & 0 & 0\\
    1 & 0 & 1 & 1 & 0 & 0 & 0 & 8\\
    1 & 1 & 0 & 1 & 1 & 1 & 0 & E\\
    1 & 1 & 1 & 1 & 0 & 1 & 1 & B\\
    \hline
  \end{tabular}
  \caption{Resumen del comportamiento de $f_1$, $f_2$, $f_3$ y $f_4$ (Parte 
  B)}
  \label{tab:b_summary}
\end{table}


\subsubsection{Implementación en Quartus}

\end{document}

