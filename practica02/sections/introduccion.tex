%% %% %% %%
%%
%% Introducción de la práctica
%%
%% %% %% %%

\documentclass[../main.tex]{subfiles}
\graphicspath{{\subfix{./img/}}}

\begin{document}
\clearpage
\section{Introducción}
Una función booleana puede representarse mediante una ecuación booleana 
compuesta por una variable binaria que identifica a la función, en seguida, se 
tiene el símbolo de igualdad y una expresión booleana; esta describe la forma 
en la cual se obtiene la salida de un sistema digital a partir de las 
variables de entrada; estas son especificadas en seguida de la variable que 
identifica la función, dentro de paréntesis y separadas por comas, 
generalmente.

La expresión booleana se entiende como una expresión algebráica conformada por 
variables binarias; es decir, sus posibles valores se encuentran acotados 
dentro del conjunto $\{0,1\}$; así como también se conforman de operadores 
lógicos y paréntesis.

Una función booleana puede tener una única salida o tener una salida múltiple.  
En caso de tener una única salida, el resultado de la función se obtiene a 
partir de evaluar las posibles combinaciones de los valores $0$ y $1$ entre 
las variables de entrada. En caso de ser una función con salida múltiple, el 
resultado se obtiene de la forma anterior, pero evaluando a su vez las 
combinaciones posibles entre las salidas de la función.

En la presente práctica, se implementan diversas funciones booleanas en la 
plataforma de desarrollo \textit{Quartus II}. A partir de la representación 
gráfica de las funciones mediante los símbolos correspondientes de las 
compuertas que conforman la función, se generan componentes que implementan 
dichas funciones, para ser empleados en otros diseños, obteniendo como ventaja 
la reducción del circuito a un símbolo que especifica únicamente las entradas 
y la salida generada, sin especificar los componentes internos.  
\end{document}



