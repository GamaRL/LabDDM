%% %% %% %%
%%
%% Introducción de la práctica
%%
%% %% %% %%

\documentclass[../main.tex]{subfiles}
\graphicspath{{\subfix{./img/}}}

\begin{document}
\clearpage
\section{Introducción}
La lógica binaria consiste en variables binarias y operaciones lógicas. Las 
variables se designan con letras del alfabeto, como $A$, $B$, $C$, $x$, $y$, 
$z$, etcétera, y cada variable tiene dos y sólo dos posibles valores: $1$ y 
$0$.  De forma general, hay tres operaciones lógicas básicas, a través de las 
cuales se pueden conformar operaciones lógicas más complejas.
\begin{itemize}
  \item \textbf{AND.} Se representa a través de un punto ($\cdot$) y su 
    resultado es $1$ si y sólo si ambas entradas son $1$.
  \item \textbf{OR.} Se representa a través de un signo más ($+$) y su 
    resultado es $0$ si y sólo si ambas entradas son $0$.
  \item \textbf{NOT.} Se representa como una testa ($\nt{a}$) y su resultado 
    es el contrario de la entrada.
\end{itemize}

Para cada combinación de los valores de variables de entrada, la definición de 
la operación lógica especifica un valor de salida. Dichas definiciones se 
pueden presentar en forma compacta con \textbf{tablas de verdad}. Una tabla de 
verdad es una tabla de todas las posibles combinaciones de las variables,  y 
muestra la relación entre los valores que las variables pueden adoptar y el 
resultado de la operación.

Durante el desarrollo de esta práctica se implementarán las compuertas lógicas 
básicas en la plataforma de desarrollo \textit{Quartus II}. Una compuerta 
lógica no es más que un circuito electrónico que opera con una o más señales 
de entrada para producir una señal de salida. En cuanto a los sistemas 
digitales, las señales eléctricas (por lo general voltajes) existen con uno de 
dos valores reconocibles.
\end{document}


