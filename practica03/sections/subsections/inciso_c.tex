%% %% %% %%
%%
%% Parte C de la práctica
%%
%% %% %% %%

\documentclass[../procedimientos.tex]{subfiles}
\graphicspath{{\subfix{../../images/}}}

\begin{document}
\clearpage
\subsection{Parte C}
\subsubsection{Propuesta}
Luis, Ana y sus padres van al cine bajo ciertas condiciones que son limitadas 
por la Madre y el Padre. Ellos van al cine si la Madre y el Padre están 
totalmente de acuerdo, si los dos no lo están, entonces no pueden ir. Ahora, 
si los dos están en desacuerdo, la salida al cine debe ser resuelta por 
mayoría absoluta entre todos los integrantes de la familia. Si existe un 
empate general, la decisión será tomada por la madre. Diseñar un circuito 
digital que señalice con un diodo LED el momento cuando pueden ir al cine.

\subsubsection{Análisis}
De forma general, el sistema tiene cuatro entradas: $l$, $a$, $m$ Y $p$ (el 
voto de Luis, Ana, su Madre y su Padre respectivamente); y la salida estará 
dada por el valor de $c(lamp)$.  La Implementación de la tabla de verdad del 
problema se muestra a continuación:
\begin{table}[H]
  \centering
  \begin{tabular}{cccc|ccc}
    \hline
    $l$ & $a$ & $m$ & $p$ & $c$\\
    \hline
    0 & 0 & 0 & 0 & 0\\
    0 & 0 & 0 & 1 & 0\\
    0 & 0 & 1 & 0 & 0\\
    0 & 0 & 1 & 1 & 1\\
    0 & 1 & 0 & 0 & 0\\
    0 & 1 & 0 & 1 & 0\\
    0 & 1 & 1 & 0 & 1\\
    0 & 1 & 1 & 1 & 1\\
    1 & 0 & 0 & 0 & 0\\
    1 & 0 & 0 & 1 & 0\\
    1 & 0 & 1 & 0 & 1\\
    1 & 0 & 1 & 1 & 1\\
    1 & 1 & 0 & 0 & 0\\
    1 & 1 & 0 & 1 & 1\\
    1 & 1 & 1 & 0 & 1\\
    1 & 1 & 1 & 1 & 1\\
    \hline
  \end{tabular}
  \caption{Tabla de verdad del problema (Sección C)}
  \label{tab:tv_c}
\end{table}

Con lo anterior, se puede deducir la forma canónica \textit{SOP}, tal como se 
muestra a continuación:
\begin{equation*}
  c(lamp) = \sum_m (3, 6, 7, 10, 11, 13, 14, 15)
\end{equation*}

Entonces, reduciendo la función lógica, se tiene que:
\begin{align*}
  c(lamp) &= \nt{l}\nt{a}mp + \nt{l}am\nt{p} + \nt{l}amp  + l\nt{a}m\nt{p} + 
  l\nt{a}mp + la\nt{m}p + lam\nt{p} + lamp\\
  &= (\nt{l}\nt{a}mp + l\nt{a}mp) + \nt{l}am\nt{p} + la\nt{m}p + (lam\nt{p} + 
  l\nt{a}m\nt{p}) + (lamp + \nt{l}amp)\\
  &= \nt{a}mp + \nt{l}am\nt{p} + la\nt{m}p + lm\nt{p} + amp\\
  &= (\nt{a}mp + amp) + \nt{l}am\nt{p} + la\nt{m}p + lm\nt{p}\\
  &= mp + \nt{l}am\nt{p} + la\nt{m}p + lm\nt{p}\\
  &= (mp + lm\nt{p}) + \nt{l}am\nt{p} + la\nt{m}p\\
  &= m(p + l\nt{p}) + \nt{l}am\nt{p} + la\nt{m}p\\
  &= m(p + l)(p + \nt{p}) + \nt{l}am\nt{p} + la\nt{m}p\\
  &= m(p + l)(1) + \nt{l}am\nt{p} + la\nt{m}p\\
  &= mp + lm + \nt{l}am\nt{p} + la\nt{m}p\\
  &= (mp + \nt{l}am\nt{p}) + (lm + la\nt{m}p)\\
  &= m(p + \nt{l}a\nt{p}) + l(m + a\nt{m}p)\\
  &= m(p + \nt{l}a)(p + \nt{p}) + l(m + ap)(m + \nt{m})\\
  &= m(p + \nt{l}a)(1) + l(m + ap)(1)\\
  &= m(p + \nt{l}a) + l(m + ap)\\
  &= m(p + \nt{l}a + l) + lap\\
  &= m(p + (\nt{l} + l)(a + l)) + lap\\
  &= m(p + (1)(a + l)) + lap\\
  &= m(p + a + l) + lap
\end{align*}
\begin{equation*}
  \boxed {
    \therefore c(lamp) = m (l + a + p) + lap
  }
\end{equation*}

De forma general, lo anterior nos indica que si la Mamá quiere ir al cine y al 
menos alguien más también quiere ir, entonces irán al cine; la única otra 
forma de ir al cine es cuando tanto Luis, Ana y el Papá quieren ir.

\subsubsection{Implementación en Quartus}

\end{document}

