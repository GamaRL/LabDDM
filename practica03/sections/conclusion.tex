%% %% %% %%
%%
%% Conclusiones de la práctica
%%
%% %% %% %%

\documentclass[../main.tex]{subfiles}
\graphicspath{{\subfix{./img/}}}

\begin{document}
\clearpage
\section{Conclusiones}
\paragraph{Ríos Lira, Gamaliel.} Con el desarrollo de la práctica, se cumplió 
el objetivo de la misma. Se pusieron en práctica los diversos elementos vistos 
durante las prácticas anteriores ---utilización de compuertas, creación de 
símbolos, utilización de vectores, etiquetas en los cables y realización de 
simulaciones funcionales--- y varios conceptos vistos durante la clase de 
teoría ---formas canónicas, reducción a través de Álgebra Booleana, 
abstracción de enunciados, etc. Además, aprendimos a cargar programas dentro 
de la tarjeta de desarrollo \textit{DE10-Lite}, a través de la cual se pasa a 
una implementación física de los sistemas digitales diseñados con la cual 
podemos interactuar directamente. Es importante la utilización de esta 
herramienta ya que será nuestra herramienta de trabajo en las prácticas 
posteriores.

\paragraph{Vélez Grande, Cinthya.} Con el desarrollo de la práctica se cumplió 
el objetivo, ya que se resolvieron diversos problemas para los cuales se 
desarrollaron  sistemas digitales que cumplen con el comportamiento esperado; 
para ello se realizó el mismo procedimiento visto con anterioridad, generando 
las respectivas tablas de verdad y planteando la función que generan las 
salidas de interés para ciertas entradas dadas; posterior a lo cual, se 
crearon los circuitos lógicos con ayuda de las herramientas gráficas 
proporcionadas por el entorno de desarrollo \textit{Quartus II}; al realizar 
las simulaciones se constataron los resultados.  Finalmente, cabe mencionar el 
hecho de que se llevaron los programas a la tarjeta de desarrollo, lo cual 
permitió una interacción directa con los circuitos creados.
\end{document}

