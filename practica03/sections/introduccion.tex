%% %% %% %%
%%
%% Introducción de la práctica
%%
%% %% %% %%

\documentclass[../main.tex]{subfiles}
\graphicspath{{\subfix{./img/}}}

\begin{document}
\clearpage
\section{Introducción}
Una función booleana, originada a partir de un problema de estudio, puede 
llevar a la construcción de un circuito lógico generado a partir de la 
conexión de múltiples compuertas lógicas. Dicho circuito presentará la misma 
cantidad de entradas y salidas planteadas para la función que le da origen, y 
debe reflejar el comportamiento descrito por la función. En el caso particular 
de presentarse una función en la que se tienen diferentes comportamientos u 
opciones dependiendo del valor de un bit a la entrada, debe optarse por 
incluir dicho bit en la cadena de entrada, evaluar los posibles casos en una 
tabla de verdad y formular la función a partir de aquellos casos que resulten 
de interés. Por ejemplo, esto se presenta en un sistema digital que tiene la 
opción de conversión de un código binario a código Gray y de código Gray a 
binario, dependiendo del estado de un bit de control ``c''; caso que se 
desarrolla a profundidad en la presente práctica, por lo cual se requieren 
conocimientos en torno a dicho sistema numérico.

El código Gray es un código sin peso y no aritmético cuya característica 
principal es que solo varía en un bit de una secuencia de dígitos a la 
siguiente; razón por la cual se le designa como código progresivo, y al 
suceder dicha progresión entre la primera y última combinación, se le 
denomina a la vez como un código cíclico.

Existen ciertas reglas que permiten realizar la conversión de sistema binario 
a código Gray y viceversa. Para el primero de los casos se tiene lo 
siguiente:

\begin{enumerate}
  \item El bit más significativo se mantiene.
  \item De izquierda a derecha se realiza un suma binaria para cada par 
    adyacente de bits y el acarreo se descarta.
\end{enumerate}

Para convertir de código Gray a binario, se realiza lo siguiente:
\begin{enumerate}
  \item El bit más significativo se mantiene.
  \item A cada uno de los bits obtenidos (del número binario) se le suma (en 
    binario) el siguiente bit adyacente del código Gray y se descarta el acarreo.
\end{enumerate}

Este código tiene diversas aplicaciones particularmente en el ámbito de los 
sistemas digitales, debido a que este permite la disminución de errores en 
sistemas cuya susceptibilidad a tenerlos aumenta con el número de cambios de 
bits entre números adyacentes; por ejemplo, evita las salidas erróneas de los 
conmutadores electromecánicos y facilita la corrección de errores en sistemas 
de comunicación digitales.

En la presente práctica se construyeron diversos sistemas digitales, los 
cuales se implementaron dentro del ambiente gráfico ofrecido por el entorno de 
desarrollo \textit{Quartus II}, para analizar el comportamiento de cada uno de 
estos y contrastar los resultados obtenidos de manera teórica con los 
simulados.
\end{document}



