%% %% %% %%
%%
%% Parte B de la práctica
%%
%% %% %% %%

\documentclass[../procedimientos.tex]{subfiles}
\graphicspath{{\subfix{../../images/}}}

\begin{document}
\clearpage
\subsection{Parte B}
\begin{em}
  Constuir un divisor de frecuencias variable para ciclos de trabajo:
  \begin{itemize}
      \item 8\%
      \item 15\%
      \item 37\%
      \item 66\%
      \item 88\%
  \end{itemize}

  Considere un periodo de $350 [ms]$.
\end{em}

Para realizar este programa, primero se hicieron una serie de cálculos.  
Tomando $f_1 = 50 [MHz]$ y $T_2 = 350 [ms]$, entonces, se tiene que:
\[ f_2 = \frac{1}{T_2} = \frac{1}{350 [ms]} \approx 2.85714285714286 [Hz]\]

Con lo anterior, se tiene que:
\[ n = \frac{f_1}{f_2} = \frac{50 [MHz]}{2.85714285714286 [Hz]} = 17500000 \]

Con lo anterior, se requeiren contar $17500000$ ciclos de $50 [MHz]$ para 
poder generar una iteración de una señal de $350 [ms]$. Ahora, tenemos que ver 
cuál será el ciclo de trabajo.

Por facilidad, la señalización se supone comienza en bajo. Por ello, es más 
cómodo contar el tiempo en el que se mantiene abajo ($p$) y no el tiempo que 
se mantiene arriba $q$.  Si se sabe que:
\[ d = \frac{q}{n} \Rightarrow q = n \cdot d \Rightarrow p = (1 - d) n\]

Con lo que se tiene:
\begin{itemize}
  \item $p_{(8\%)}  = (1 - 0.08) 17500000 = 16100000$
  \item $p_{(15\%)} = (1 - 0.15) 17500000 = 14875000$
  \item $p_{(37\%)} = (1 - 0.37) 17500000 = 11025000$
  \item $p_{(66\%)} = (1 - 0.66) 17500000 =  5950000$
  \item $p_{(88\%)} = (1 - 0.88) 17500000 =  2100000$
\end{itemize}

Con estos valores, se procedió a crear una entidad general para todas, la cual 
a través de un parámetro se pudiera configurar. La entidad creada fue:
\begin{lstlisting}[language=vhdl]
library ieee;
use ieee.std_logic_1164.all;

entity DivFreq is
	generic(off_time: integer);
	port(
		reloj: in std_logic;
		clk:	out std_logic
	);
end entity;

architecture Behave of DivFreq is
	constant max: integer := 17500000;
	signal count: integer range 0 to max;
begin
	process
	begin
		wait until reloj'event and reloj = '1';
		
		if count < max then
			count <= count + 1;
		else
			count <= 0;
		end if;
		
		if count < off_time then
			clk <= '0';
		else
			clk <= '1';
		end if;
	end process;
end;
\end{lstlisting}

Tal como se puede ver, se tiene el parámetro \texttt{off\_time}, a través del 
cual se le indica el número de ciclos que debe mantenerse apagado. Esto se 
puede integrar a través de un multiplexor 4:1, tal como se muestra a 
continuación:
\begin{lstlisting}[language=vhdl]
library ieee;
use ieee.std_logic_1164.all;

entity Mux8_1 is
	port(
		n: in std_logic_vector(7 downto 0);
		s:	in std_logic_vector(2 downto 0);
		o: out std_logic
	);
end entity;

architecture Behave of Mux8_1 is
begin
	with s select
	o <=
		n(0) when "000",
		n(1) when "001",
		n(2) when "010",
		n(3) when "011",
		n(4) when "100",
		n(5) when "101",
		n(6) when "110",
		n(7) when "111";
end;
\end{lstlisting}

Finalmente, todo se unió en una entidad adicional:
\begin{lstlisting}[language=vhdl]
library ieee;
use ieee.std_logic_1164.all;

entity Ejercicio is
	port(
		reloj:	in std_logic;
		f:	in std_logic_vector(2 downto 0);
		o: out std_logic
	);
end entity;

architecture Behave of Ejercicio is
	component DivFreq is
		generic(off_time: integer);
		port(
			reloj: in std_logic;
			clk:	out std_logic
		);
	end component;
	
	component Mux8_1 is
		port(
			n: in std_logic_vector(7 downto 0);
			s:	in std_logic_vector(2 downto 0);
			o: out std_logic
		);
	end component;
	
	signal s: std_logic_vector(7 downto 0);
begin
	df0 :DivFreq
		generic map(off_time => 16100000)
		port map(
			reloj => reloj,
			clk => s(0)
		);
	
	df1 :DivFreq
		generic map(off_time => 14875000)
		port map(
			reloj => reloj,
			clk => s(1)
		);
	
	df2 :DivFreq
		generic map(off_time => 11025000)
		port map(
			reloj => reloj,
			clk => s(2)
		);
	
	df3 :DivFreq
		generic map(off_time => 5950000)
		port map(
			reloj => reloj,
			clk => s(3)
		);
	
	df4 :DivFreq
		generic map(off_time => 2100000)
		port map(
			reloj => reloj,
			clk => s(4)
		);
	
	mux :Mux8_1 port map(
		n => s,
		s => f,
		o => o
	);
end;
\end{lstlisting}

De forma general, se puede ver que se constroyeron cinco divisores de 
frecuencia (con los parámetros calculados anteriormente). De estas señales, 
únicamente se selecciona una señal a través el multiplexor 4:1 y se manda a la 
salida de la entidad.
\end{document}
