%% %% %% %%
%%
%% Previo de la práctica
%%
%% %% %% %%

\documentclass[../procedimientos.tex]{subfiles}
\graphicspath{{\subfix{../../images/}}}

\begin{document}
\subsection{Previo} \label{subs:previo}
\subsubsection*{Pregunta 1}
\begin{em}
  ¿Cómo se utiliza un contador como divisor de frecuencias?
\end{em}

Se necesitan los siguientes elementos:
\begin{itemize}
  \item \textbf{Contador de n bits.} El contador de n bits es el encargado de 
    realizar el conteo de ciclos de reloj de $CLK$.
  \item \textbf{Flip-Flop D.} El Flip-Flop tipo D es el encargado de almacenar 
    el estado de la bandera encargada de cambiar el estado de la señal de salida 
    $CLK_{div}$.
  \item \textbf{Comparador de Igualdad.} Cuando el contador de n bits alcance 
    el número máximo de conteo ($Count_{MAX}$), se activará una bandera para 
    cambiar de estado al Flip-Flop tipo D, el encargado de activar dicha 
    bandera es el comparador de igualdad.
  \item \textbf{Sumador.} Este componente será el encargado de incrementar en 1 el contador de n 
    bits hasta el rango requerido por la aplicación.
\end{itemize}

La cuestión es encontrar el número de ciclos de una frequencia base que 
completan la frecuencia esperada. El contador cuenta el número de flancos 
ascendentes o descendentes de la señal y con ello, se detecta cuando se 
tiene un ciclo completado (de la nueva señal).

\subsubsection*{Pregunta 2}
\begin{em}
  ¿Qué es un PWM, en que basa su funcionamiento?
\end{em}

\textit{PWM} son las siglas de \textit{Pulse Width Modulation} (Modulación por 
ancho de pulso). Es una técnica que se usa para transmitir señales analógicas 
cuya señal portadora será digital. En esta técnica se modifica el ciclo de 
trabajo de una señal periódica (una senoidal o una cuadrada, por ejemplo), ya 
sea para transmitir información a través de un canal de comunicaciones o para 
controlar la cantidad de energía que se envía a una carga.

Básicamente, consiste en activar una salida digital durante un tiempo y 
mantenerla apagada durante el resto, generando así pulsos positivos que se 
repiten de manera constante. Por tanto, la frecuencia es constante (es decir, 
el tiempo entre disparo de pulsos), mientras que se hace variar la anchura del 
pulso, el \textit{duty cycle}. El promedio de esta tensión de salida, a lo 
largo del tiempo, será igual al valor analógico deseado.

\subsubsection*{Pregunta 3}
\begin{em}
  ¿Qué es un atributo a una señal?
\end{em}

Algunas d elas características que caracterizan a una señal son:
\begin{itemize}
  \item \textbf{Periodo.} El tiempo que tarda la señal en completar un ciclo.
  \item \textbf{Frecuencia.} El número de ciclos que se completan en una 
    unidad de tiempo.
  \item \textbf{Amplitud.} Es la máxima distancia de un punto de equilibrio a 
    la un punto de la señal.
\end{itemize}

\subsubsection*{Pregunta 4}
\begin{em}
  ¿Qué es una variable y una constante? ¿Cómo se inicializa una variable a un 
  valor (asignación)?
\end{em}

\paragraph{Constantes.} Es un elemento que se inicializa con un valor 
determinado, el cual no puede ser modificado, es decir siempre conserva el 
mismo valor. Esto se realiza con la palabra reservada \texttt{CONSTANT}.
\begin{lstlisting}[language=vhdl]
-- Declaracion de una constante
constant e : integer := 10;
\end{lstlisting}

\paragraph{Variables.} Es lo mismo que una constante, pero con la diferencia 
que puede ser modificada en cualquier instante, aunque también es posible 
inicializarlas. La palabra reservada \texttt{VARIABLE} es la que permite 
declarar variables.
\begin{lstlisting}[language=vhdl]
-- Declaracion de una variable
variable e : std_logic(3 downto 0) := "1010";
\end{lstlisting}

\subsubsection*{Pregunta 5}
\begin{em}
  ¿Qué es y cómo se define un ciclo de trabajo en una señal astable?
\end{em}

El ciclo de trabajo (\textit{duty cycle}) de una señal periódica es el ancho 
de su parte positiva, en relación con el período. Está expresado en 
porcentaje, por tanto, un \textit{duty cycle} ($d$) de 10\% indica que está 10 
de 100 a nivel alto.

\[ d = \frac{q}{T} \]

\begin{itemize}
  \item $q$: Tiempo en parte positiva
  \item $T$: Periodo, tiempo total
\end{itemize}

\end{document}

