%% %% %% %%
%%
%% Parte C de la práctica
%%
%% %% %% %%

\documentclass[../procedimientos.tex]{subfiles}
\graphicspath{{\subfix{../../images/}}}

\begin{document}
\clearpage
\subsection{Parte C}
\subsubsection*{Propuesta}
\begin{em}
  Se proopone hacer un divisor de frecuencias complementario. Para esto, se 
  considerará un periodo de $350 [ms]$ (tal como en la actividad anterior).
\end{em}

\subsection*{Solución}
Para realizar dos señales que se complementen, se necesitan tener ciclos de 
trabajo complementarios: Por ejemplo, se puede tener $d_1 = 5\%$ y $d_2 = 
95\%$. De esta forma, cuando una señal esté prendida, la otra estará apagada 
(alternando este comportamiento entre ambas).

Con los cálculos de la sección anterior, se tiene:
\[ n = 17500000 \]

Calculando los valores de $q$, se tiene que:
\begin{itemize}
  \item $p_{(5\%)}  = (1 - 0.05) 17500000 = 16625000$
  \item $p_{(95\%)} = (1 - 0.95) 17500000 =   875000$
\end{itemize}

Con lo anterior, únicamente, se tiene que contruir una entidad con ambas 
salidas:
\begin{lstlisting}[language=vhdl]
library ieee;
use ieee.std_logic_1164.all;

entity ParteC is
	port(
		clk50MHz:	in std_logic;
		clk05: 		out std_logic;
		clk95: 		out std_logic
	);
end entity;

architecture Behave of ParteC is
	component DivFreq is
		generic(off_time: integer);
		port(
			reloj: in std_logic;
			clk:	out std_logic
		);
	end component;
	
	component Mux8_1 is
		port(
			n: in std_logic_vector(7 downto 0);
			s:	in std_logic_vector(2 downto 0);
			o: out std_logic
		);
	end component;
	
	signal s: std_logic_vector(7 downto 0);
begin
	df0 :DivFreq
		generic map(off_time => 16625000)
		port map(
			reloj => clk50MHz,
			clk => clk05
		);
	
	df1 :DivFreq
		generic map(off_time => 875000)
		port map(
			reloj => clk50MHz,
			clk => clk95
		);
end;
\end{lstlisting}

Al llevarlo a la tarjeta, se observa que mientras un LED enciende un corto 
periodo de tiempo (permanece apagado mucho tiempo), el otro LED enciede un 
largo periodo de tiempo (permanece apagado poco tiempo). Y justamente, 
mientras uno de los LED se enciende, el otro se apaga.

\end{document}

