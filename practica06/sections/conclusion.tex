%% %% %% %%
%%
%% Conclusiones de la práctica
%%
%% %% %% %%

\documentclass[../main.tex]{subfiles}
\graphicspath{{\subfix{./img/}}}

\begin{document}
\clearpage
\section{Conclusiones}
\paragraph{Ríos Lira, Gamaliel.} Con la realización de la prácticas, se 
completó satisfactoriamente el objetivo planteado al comienzo de la misma. Se 
comprendió cómo utlizar un contador para realizar un divisor de frecuencias.  
Con esto, es posible manejar procesos a un nivel más apto para el 
entendimiento humano (frecuencias pequeñas y periodos grandes). De no ser de 
esta forma, los sitemas que tendríamos ejecutarían millones de veces por 
segundo sin que nosotros nos diéramos cuenta.

La implementación a través del lenguaje \textit{VHDL} me pareció muy sencilla 
en comparación con la implementación directamente digital, lo cual representa 
una gran ventaja. La única limitante que encuentro es que el reloj de $50 
[MHz]$ oscila muy rápido y se necesitan contadores muy grandes para obtener el 
divisor de frecuencias.

Finalmente, no agregué algún mecanismo de comprobación de las señales 
solicitadas ya que no es posible agregar el comportamiento de la FPGA a través 
de algún tipo de foto y una simulación no es factible por la catidad de 
oscilaciones que se requieren para completar un ciclo. Aún así, el código se 
adjunta para su comprobación en la tarjeta ---de ser necesario---.
%\paragraph{Vélez Grande, Cinthya.}
\end{document}

