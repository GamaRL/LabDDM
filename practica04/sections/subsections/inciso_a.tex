%% %% %% %%
%%
%% Parte A de la práctica
%%
%% %% %% %%

\documentclass[../procedimientos.tex]{subfiles}
\graphicspath{{\subfix{../../images/}}}

\begin{document}
\clearpage
\subsection{Parte A}
\subsubsection{Instrucciones}
\begin{em}
  Programar las funciones mostradas, en un formato SOP y POS mínimo (reducido) 
  en forma flujo de datos, para su posterior simulación dentro de la 
  plataforma \textit{Quartus II}.
  \begin{itemize}
    \item \textbf{Mediante POS}
      \begin{equation*}
        f(xyzt) = (x + \n{x}y + \overline{yt})(x + \overline{xy}\cdot z)
      \end{equation*}
    \item \textbf{Mediante SOP}
      \begin{equation*}
        f(uvtw) = vt(u+t\n{w})(u+\n{w})+wu(v+t)
      \end{equation*}
  \end{itemize}
\end{em}

Primero, se desarrollará la función $f(xyzt)$ a través de álgebra booleana 
para encontrar una forma POS reducida:
\begin{align*}
  f(xyzt) &= (x + \n{x}y + \overline{yt})(x + \overline{xy}\cdot z)\\
  &= (\cancel{(x+\n{x})}(x+y) + \overline{yt})(x + (\n{x}+\n{y})z)\\
  &= \cancel{(x+y + \n{y} + \n{t})}(x + (\n{x}+\n{y})z)\\
  &= \cancel{(x+\n{x}+\n{y})}(x+z)\\
  &= (x+z)
\end{align*}
\begin{equation*}
  \boxed{
    \therefore f(xyzt) = (x+z)
  }
\end{equation*}

Posteriormente, se desarrollará la función $f(wvtw)$ a través de álgebra 
booleana para encontrar una forma SOP reducida:
\begin{align*}
  f(uvtw) &= vt(u+t\n{w})(u+\n{w})+wu(v+t)\\
  &= vt(u+t)(u+\n{w})\cancel{(u+\n{w})}+wu(v+t)\\
  &= vt(u+t)(u+\n{w})+wu(v+t)\\
  &= vt(u+t\n{w})+wuv+wut\\
  &= uvt+vt\n{w}+uvw+utw
\end{align*}
\begin{equation*}
  \boxed{
    \therefore f(uvtw) = uvt+vt\n{w}+uvw+utw
  }
\end{equation*}

\subsubsection{Análisis}\label{subs:analisis_a}
\subsubsection{Implementación en Quartus}\label{subs:a_imp}
\subsubsection{Ejecución en la FPGA}
\end{document}

