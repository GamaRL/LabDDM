%% %% %% %%
%%
%% Introducción de la práctica
%%
%% %% %% %%

\documentclass[../main.tex]{subfiles}
\graphicspath{{\subfix{./img/}}}

\begin{document}
\clearpage
\section{Introducción}
El lenguaje \textit{VHDL} (\textit{Very High Speed Integrates Circuit Hardware 
Description Language}) es un lenguaje de descripción de circuitos electrónicos 
digitales que utiliza distintos niveles de abstracción. \textit{VHDL} nació 
como una manera estándar de documentar circuitos; desarrollado por el gobierno 
de Estados Unidos en la década de los 80’s.

De manera particular, es un lenguaje que permite tanto una descripción de la 
estructura del circuito (descripción a partir de subcircuitos más sencillos), 
como la especificación de la funcionalidad del circuito utilizando formas 
familiares a los lenguajes de programación.

Los circuitos descritos en \textit{VHDL} pueden ser simulados utilizando 
herramientas que reproducen el funcionamiento del circuito descrito; la misión 
más importante de un lenguaje de descripción HW es que sea capaz de simular 
perfectamente el comportamiento lógico de un circuito sin que el programador 
necesite imponer restricciones.

Un circuito o subcircuito descrito mediante \textit{VHDL} se denomina diseño 
de entidad (design entity). Está compuesto por dos partes: la declaración de 
la entidad, entity, (donde se declaran las señales de entrada y salida, por lo 
tanto es el modelo de interfaz con el exterior) y la arquitectura, 
architecture (donde se definen los detalles del circuito, es decir, es la 
especificación del funcionamiento de una entidad).

Por otro lado están las bibliotecas, donde se almacenan los elementos de 
diseño: tipos de datos, operadores, componentes, funciones, etc...
Hay dos bibliotecas que siempre son visibles por defecto: std (la estándar) y 
work (la de trabajo) y que no es necesario declarar.

Los elementos de las bibliotecas se organizan en paquetes o bibliotecas 
(Packages) y hay que declararlos para poder utilizarlos dentro de la 
construcción de los circuitos.

A lo largo de la presente práctica se han implementado, a través del lenguaje 
\textit{VHDL}, diversos circuitos que se diseñaron previamente mediante 
herramientas gráficas proporcionadas por el entorno de desarrollo; como se 
verá, es posible obtener equivalencias en los resultados mediante ambas 
implementaciones.
\end{document}

